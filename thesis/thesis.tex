\documentclass[10pt]{article}
\usepackage{hyperref}
\usepackage{textcomp, xspace}
\usepackage[fleqn]{amsmath}
\usepackage{amsfonts}
\usepackage{cancel}
\usepackage{graphicx}
\usepackage{accents}
\begin{document}

\title{Master's Thesis}
\author{Rohit Gupta, Sumit Basu}

\maketitle

\begin{abstract}
Fill it.
\end{abstract}

\section{Introduction}
 
\section{Motivation}

\section{1D simulation}
Bamboo is a natural composite which is composed of fibers embedded in a matrix of parenchyma cells. Bamboo fibers are mainly composed of cellulose, hemicellulose and lignin [L.Y.Mwaikambo et al.]. Fibers are spread out across the cross-section in a graded manner with higher density towards the periphery. Also, the size of parenchyma cells decreases along the radially outward direction as the air content reduces. This results in axisymmetric areal density variation in the radial direction. [Plot showing the distribution of fibers.]\par 
This can be modeled as the distribution of two materials, and air (capture in parenchyma cells). First material being the denser and stiffer fibers and second material being the parenchyma cellular material excluding air. The properties of fibers and parenchyma are taken from [Mannan et al., L.Y.Mwaikambo et al.]. \par
Bamboo, like any other living organism, is a result of an evolutionary process. Survival of the fittest means that bamboo species is nature's best solution for some natural condition lead to the existence of bamboo. Bamboo grows tall up to 20m to rise above the other competing plantation. With such a slender structure, bending load due to high-speed tropical winds are a significant constraint which the evolution had overcome. Bamboo has a very high specific strength, which is also the desired attribute in industrial applications. It is desired to develop composites with high stiffness and lower weight.\par 
Therefore, we frame it as a constrained maximization problem, with the objective function being specific strength, optimizing the radial distribution of fibers and parenchyma. The dimensions of the composites are kept the same as in [Mannan et al.]. Also, the limits of max bending moment and stress are taken from [Mannan et al.] 
\begin{equation*}
\begin{aligned}
& \underset{r}{\text{maximize}}
& & strength(r) \\
& \text{subject to}
& & stress\,at\,every\, point \leq max\, stress\\
& & & bending\, moment \leq max\, moment
\end{aligned}
\end{equation*}

Now, specific flexural rigidity is used as measure of strength which can be written as 
\begin{equation}
\begin{split}
strength(x) =  &\frac{EI}{\rho}\\
=&\frac{\sum^{r_o}_{r_i} E(r)I(r)}{\sum^{r_o}_{r_i}\rho(r)}
\end{split}
\end{equation}
\lsubsection{Optimization Problem}
\begin{equation}
\begin{split}
max \quad& \frac{\sum^{r_o}_{r_i} E(r)I(r)}{\sum^{r_o}_{r_i}\rho(r)}\\
s.t. \quad& \frac{rE(r)}{R}\leq \sigma_{max} \quad \forall r \in [r_i, r_o]\\
& \frac{EI}{R}\leq M_{max}
\end{split}
\end{equation}
where $E(r) = \nu_1(r) E_1 + \nu_2(r) E_2$, is the Young's Modulus obtained using rule of mixtures[ at a radial distance $r$ from the center. 
\section{Method}
The idea of asymptotic homogenization.
In a repeating cell Y,
\begin{equation}
	\label{First}
	\sigma_{ij} = C_{ijkl}\epsilon_{kl}	
\end{equation}
where $C_{ijkl}(\underaccent{\tilde}{x}+\underaccent{\tilde}{u}\underaccent{\tilde}{Y})=C_{ijkl}(\underaccent{\tilde}{x})$
\begin{equation}
\Rightarrow C_{ijkl}(x_1+n_1Y_1\, x_2+n_2Y_2\,x_3+n_3Y_3) = C_{ijkl}(x_1,x_2,x_3)
\end{equation}
$C_{ijkl}(\underaccent{\tilde}{x}) $ is Y-periodic
\begin{eqnarray}
&\underaccent{\tilde}{y} = \frac{\underaccent{\tilde}{x}}{\epsilon}\\
&\Rightarrow g = g(\underaccent{\tilde}{x},\frac{\underaccent{\tilde}{x}}{\epsilon}) = g(\underaccent{\tilde}{x} \,\underaccent{\tilde}{y})
\end{eqnarray}
$\underaccent{\tilde}{x} = (x_1, x_2, x_3) \in \mathbb{R}^3$ defines the domain of the composite $\Omega$. The domain is composed of base cells of dimensions, $\varepsilon Y_1 , \varepsilon Y_2,\varepsilon Y_3$ where $\underaccent{\tilde}{y}=\frac{\underaccent{\tilde}{x}}{\varepsilon}$
\subsection{1D Elasticity}
\begin{eqnarray}
&\sigma^\varepsilon = E^\varepsilon\frac{\partial u^\varepsilon}{\partial x}\\
&\frac{\partial \sigma^\varepsilon}{\partial x}+\gamma^\varepsilon=0 & E^\varepsilon \, \gamma^\varepsilon \rightarrow macroscopically uniform
\end{eqnarray}
Inside each cell, 
\begin{eqnarray}
E^\varepsilon (x, \frac{x}{\varepsilon})&=E(y) \\ 
\gamma^\varepsilon(x, \frac{x}{\varepsilon})&=\gamma(y)
\end{eqnarray} 
Let
\begin{eqnarray}
u^\varepsilon(x)=u^0{x,y}+\varepsilon u^1(x,y)+\varepsilon^2 u^2(x,y)+ ...\\
\sigma^\varepsilon(x)=\sigma^0{x,y}+\varepsilon \sigma^1(x,y)+\varepsilon^2 \sigma^2(x,y)+ ...
\end{eqnarray}


\subsection{Optimal Design of Elastic structures}

\begin{center}
$\textbf{b} \rightarrow$ body forces\\
$\textbf{t} \rightarrow$ surface tractions
\end{center}

Optimal choice of $\mathbb{C}_{ijkl} \in U_{ad} \leftarrow $ admissible set of elasticity \par $\mathbb{C}_{ijkl}(\textbf{x}) \forall \textbf{x} \in \Omega $ has 21 independent components \par $a_E(\textbf{u}, \textbf{v}) = \int_\Omega \mathbb{C}_{ijkl}\,\varepsilon_{kl}(\textbf{u})\,\varepsilon_{kl}(\textbf{v})d\textbf{v} \rightarrow $ energy bilinear form \par
$L(\textbf{v}) = \int_\Omega \textbf{v}\, d\textbf{x}+\int_{\partial\Omega_t} \textbf{t}\cdot\textbf{v}ds \rightarrow $ load linear form.\par
\hfill \break
Minimum compliance problem:
\begin{eqnarray}
minimize & L(\textbf{v}),\\
subject\, to & \mathbb{C}_{ijkl} \in \mathbb{U}_{ad}\\
		  & a_E(\textbf{u}, \textbf{v}) = L(\textbf{v}) &\forall \textbf{v} \in \mathbb{U} 
\end{eqnarray}
where $\mathbb{U}\rightarrow $ kinematically admissible displacements.\\
For optimal shape design:
\begin{eqnarray}
\mathbb{C}_{ijkl}(\textbf{x}) = \chi(\textbf{x})\overline{\mathbb{C}}_{ijkl}, & \textrm{where  } \overline{\mathbb{C}}_{ijkl}\rightarrow\textrm{stiffness matrix of the material}\\
\chi(\textbf{x}) =
    \begin{cases}
        1 & \text{if $\textbf{x}\in \Omega^m$,}\\
        0 & \text{if $\textbf{x}\in \Omega\backslash\Omega^m$}
    \end{cases}
\end{eqnarray}
where $\Omega^m \rightarrow$ part of the domain occupied by the material.\\
For sizing problem:
\begin{eqnarray}
\mathbb{C}_{ijkl}(\textbf{x}) = h(\textbf{x})\overline{\mathbb{C}}_{ijkl}\\
& \int_\Omega \chi(\textbf{x})d\textbf{x}=V_f\\
\& & \int_\Omega h(\textbf{x})d\textbf{x}=V_f.
\end{eqnarray}
where $h(x)$ is a sizing function. \hfill
\break
Traditionally shape design problems are initiated in the following manner:
\begin{align}
Ref\, doamin: & \Omega_0\in \mathbb{R}^3\\
\underline{\phi}:  & \Omega_0 \rightarrow \phi(\Omega_0) \text{is a diffeomorphism.}\\
L(\textbf{v})&=\int_{\Omega_0} \textbf{f}\cdot\textbf{v} |det(D\underline{\phi}^{-1})|d\textbf{x}+\int_{\partial\Omega_t} \textbf{t}\cdot\textbf{v}|det(D\underline{\phi}^{-1})|ds
\end{align}

\begin{equation}
\begin{split}
a_E &=\int_\Omega \mathbb{C}_{ijkl}(\textbf{x}\varepsilon_{kl}(\textbf{v})\varepsilon_{ij}(\textbf{v})d\textbf{x}\\
&=\int_{\Omega_0} \mathbb{C}_{ijkl}\varepsilon_{kl}(\textbf{v})\varepsilon_{ij}(\textbf{v})|det(D\underline{\phi}^{-1})|d\textbf{x}
\end{split}
\end{equation}

Now,
\begin{equation}	\label{compliance}
\begin{split}
\mathbb{C}_{ijkl} \varepsilon_{kl} &= \mathbb{C}_{ijkl}\frac{1}{2}(u_{k,l}+u_{l,k})\\
 & =\frac{1}{2}\mathbb{C}_{ijkl}u_{k,l}+\frac{1}{2}\mathbb{C}_{ijlk}u_{l,k}\\
 & =\mathbb{C}_{ijkl}u_{k,l}
\end{split}
\end{equation}
\begin{equation}
\begin{split}
a_E &= \int_{\Omega_0}\mathbb{C}_{ijkl}u_{k,l}(\textbf{u})u_{i,j}(\textbf{v})|det(D\underline{\phi}^{-1}|d\textbf{x}\\
&= \int_{\Omega_0}\mathbb{C}_{ijkl}\frac{\partial u_k}{\partial\textbf{x}_m}(D\underline{\phi}^{-1})_{ml}\frac{\partial u_i}{\partial \textbf{x}_p}(D\phi^{-1})_{pj}|det(D\underline{\phi}^{-1})|d\textbf{x}\\
\end{split}
\end{equation}
\begin{eqnarray}
\Rightarrow \mathbb{C}_{ijkl}(D\underline{\phi}^{-1})_{ml}(D\underline{\phi}^{-1})_{pj}|det(D\underline{\phi}^{-1})| = \bar{\mathbb{C}}_{ipkm}\\
\bar{\mathbb{C}}_{ijkl} = \mathbb{C}_{ipkm}(D\underline{\phi}^{-1})_{lm}(D\underline{\phi}^{-1})_{jp}|det(D\underline{\phi}^{-1})|
\end{eqnarray}
Treating $\underline{\phi}$ as a design variable is tedious.

\subsection{Homogenization method}
\begin{equation}
E_{ijkl}^\varepsilon (\textbf{x}) = E_{ijkl}(\textbf{x},\textbf{y}), \qquad \textbf{y} = \frac{\textbf{x}}{\varepsilon}
\end{equation}
The tensor $E_{ijkl}^\varepsilon$ is a material constant which satisfies the symmetry condition and is assumed to satisfy strong ellipticity condition for every $\textbf{x}$.
\begin{gather}
\Rightarrow E_{ijkl}^\varepsilon =E_{jikl}^\varepsilon =E_{ijlk}^\varepsilon =E_{klij}^\varepsilon\\
E_{ijkl}^\varepsilon(\textbf{x})\textbf{X}_{ij}\textbf{X}_{kl}\geq m\textbf{X}_{ij}\textbf{X}_{ij} \qquad \text{for some } m>0 \text{ \& } \forall \quad \textbf{X}_{ij}=\textbf{X}_{ji}
\end{gather}
Let the domain$\Omega$ has a boundary $\Gamma$. Let $\textbf{f}$ be the body force acting on $\Omega$ and $\textbf{t}$ be the traction acting on $\Gamma_t$ part of the boundary $\Gamma$. Also, let $\Gamma_D$ be the part of boundary on which displacement is defined. Then the displacement $\textbf{u}^\varepsilon$ can be obtained as the solution to the following minimization problem
\begin{align}
\label{fem}
\min_{\textbf{v}^\varepsilon\in U} \quad F^\varepsilon(\textbf{v}^\varepsilon),
\end{align} 
where $F^\varepsilon$ is total potential energy given as	
\begin{eqnarray} \label{tpe}
F^\varepsilon(\textbf{v}^\varepsilon) = \frac{1}{2}\int_\Omega E^\varepsilon_{ijkl}\varepsilon_{kl}(\textbf{v}^\varepsilon)\varepsilon_{ij}(\textbf{v}^\varepsilon)dx-\int_\Omega\textbf{f}\cdot\textbf{v}^\varepsilon dx - \int_{\Gamma_t}\textbf{t}\cdot\textbf{v}^\varepsilon ds
%F^\varepsilon(\textbf{v}^\varepsilon) = \frac{1}{2}a^\varepsilon(\textbf{v}^\varepsilon, \textbf{v}^\varepsilon)-L(\textbf{v}^\varepsilon),\\
%a^\varepsilon(\textbf{u},\textbf{v})=\int_\Omega E^\varepsilon_{ijkl}\varepsilon_{kl}(\textbf{u})\varepsilon_{ij}(\textbf{v})dx,\\
%L(\textbf{v})=\int_\Omega\textbf{f}\cdot\textbf{v}dx + \int_{\Gamma_t}\textbf{t}\cdot\textbf{v}ds
\end{eqnarray}
and $\mathcal{U}$ is the set of admissible displacements defined such that
\begin{equation}
\mathcal{U} = \{\textbf{v} = v_i\textbf{e}_i :\, v_i\in H^1(\Omega) \text{ and } \textbf{v}\in\mathcal{G} \text{ on } \Gamma_D\}
\end{equation}
where $\mathcal{G}$ is set of displacement defined along the boundary $\Gamma_D$.\\
Let 
\begin{align}
\textbf{v}^\varepsilon(\textbf{x}) = \textbf{v}_0(\textbf{x})+\varepsilon\textbf{v}_1(\textbf{x},\textbf{y}),\qquad \textbf{y}=\frac{\textbf{x}}{\varepsilon}.
\end{align}
Using chain rule for functions in two variables\\
\begin{equation}
\begin{split}
\frac{\partial f(\textbf{x}, \textbf{y(x)})}{\partial \textbf{x}} &= \frac{\partial f}{\partial \textbf{x}}+\frac{\partial f}{\partial \textbf{y}}\frac{\partial \textbf{y}}{\partial \textbf{x}}\\
&=\frac{\partial f}{\partial \textbf{x}}+\frac{1}{\varepsilon}\frac{\partial f}{\partial \textbf{y}}
\end{split}
\end{equation}
Using above two equations, we can write the linerized strain as
\begin{equation}
\begin{split}
\epsilon_{ij}(\textbf{v}^\varepsilon(\textbf{x})) &= \frac{\partial (v_{0i}(\textbf{x})+\varepsilon v_{1i}(\textbf{x},\textbf{y}))}{\partial x_j}\\
&=\frac{\partial v_{0i}}{\partial x_j}+\varepsilon\bigg\{\frac{\partial v_{1i}}{\partial x_j}+\frac{1}{\varepsilon}\frac{\partial v_{1i}}{\partial y_j}\bigg\}\\
&\approx \frac{\partial v_{0i}}{\partial x_j}+\frac{\partial v_{1i}}{\partial y_j} \qquad \leftarrow \{ \varepsilon << 1\}
\end{split}
\end{equation}
Therefore, equation \eqref{tpe} can be written as 
\begin{multline}
F^\varepsilon(\textbf{v}^\varepsilon) = \frac{1}{2}\int_\Omega E^\varepsilon_{ijkl}\bigg (\frac{\partial v_{0k}}{\partial x_l}+\frac{\partial v_{1k}}{\partial y_l}\bigg )\bigg (\frac{\partial v_{0i}}{\partial x_j}+\frac{\partial v_{1i}}{\partial y_j}\bigg )dx\\
-\int_\Omega\textbf{f}\cdot\textbf{v}_0 dx - \int_{\Gamma_t}\textbf{t}\cdot\textbf{v}_0 ds + \varepsilon R^\varepsilon(\textbf{v}_0, \textbf{v}_1)
\end{multline}
Here, $R^\varepsilon$ is the contribution of $\varepsilon\textbf{v}_1$ in the calculation of energy from body force and traction.
Using 
\begin{equation}
\lim_{\varepsilon\rightarrow 0}\int_\Omega \Phi(x, x/\varepsilon)dx = \frac{1}{|Y|}\int_\Omega\int_Y \Phi(x,y)dydx,
\end{equation}
we get,
\begin{equation}
\begin{split}
\lim_{\varepsilon\rightarrow 0}\,F^\varepsilon(\textbf{v}^\varepsilon)&=F(\textbf{v}_0,\textbf{v}_1)\\
&=\frac{1}{2|Y|}\int_\Omega\int_Y E_{ijkl}(x,y) \bigg (\frac{\partial v_{0k}}{\partial x_l}+\frac{\partial v_{1k}}{\partial y_l}\bigg )\bigg (\frac{\partial v_{0i}}{\partial x_j}+\frac{\partial v_{1i}}{\partial y_j}\bigg )dy\,dx\\
&\qquad-\int_\Omega\textbf{f}\cdot\textbf{v}_0 dx - \int_{\Gamma_t}\textbf{t}\cdot\textbf{v}_0 ds 
\end{split}
\end{equation}
A minimizer $\{\textbf{u}_0, \textbf{u}_1\}$ of the functional $F$, follow the following equations:

\begin{equation}\label{first}
\begin{aligned}
&\frac{1}{|Y|}\int_\Omega\int_Y E_{ijkl}(x,y)\bigg (\frac{\partial u_{0k}}{\partial x_l}+\frac{\partial u_{1k}}{\partial y_l}\bigg )\bigg (\frac{\partial v_{0i}}{\partial x_j}\bigg ) dy\,dx\\
&=\int_\Omega\textbf{f}\cdot\textbf{v}_0 dx + \int_{\Gamma_t}\textbf{t}\cdot\textbf{v}_0 ds \qquad \text{for every } \textbf{v}_0
\end{aligned}
\end{equation}
\begin{equation}
\frac{1}{|Y|}\int_\Omega\int_Y E_{ijkl}(x,y)\bigg (\frac{\partial u_{0k}}{\partial x_l}+\frac{\partial u_{1k}}{\partial y_l}\bigg )\bigg (\frac{\partial v_{i}}{\partial x_j}\bigg ) dy\,dx = 0, \qquad \text{for every } \textbf{v}_1
\end{equation}
Now, from localizing $u_{1k}$
\begin{equation}\label{localization}
u_{1k}(x,y)=-\chi^{pq}_k(y)\frac{\partial u_{0p}}{\partial x_q}(x),
\end{equation}
\begin{align*}
\Rightarrow &\int_\Omega\int_Y E_{ijkl}(x,y)\bigg (\frac{\partial u_{0k}}{\partial x_l}-\frac{\partial \chi^{pq}_k}{\partial y_l}\frac{\partial u_{0p}}{\partial x_q}\bigg )\frac{\partial v_{i}}{\partial x_j} dy\,dx=0\\
&\int_\Omega\int_Y \bigg (E_{ijkl}\frac{\partial u_{0k}}{\partial x_l}-E_{ijkl}\frac{\partial \chi^{pq}_k}{\partial y_l}\frac{\partial u_{0p}}{\partial x_q}\bigg )\frac{\partial v_{i}}{\partial x_j} dy\,dx=0\\
&\int_\Omega\int_Y \bigg (E_{ijkl}\frac{\partial u_{0k}}{\partial x_l}-E_{ijpq}\frac{\partial \chi^{kl}_p}{\partial y_q}\frac{\partial u_{0k}}{\partial x_l}\bigg )\frac{\partial v_{i}}{\partial x_j} dy\,dx=0\\
&\int_\Omega\int_Y \frac{\partial u_{0k}}{\partial x_l}\bigg (E_{ijkl}-E_{ijpq}\frac{\partial \chi^{kl}_p}{\partial y_q}\bigg )\frac{\partial v_{i}}{\partial x_j} dy\,dx=0\\
&\int_\Omega\frac{\partial u_{0k}}{\partial x_l}dx\cdot\int_Y \bigg (E_{ijkl}-E_{ijpq}\frac{\partial \chi^{kl}_p}{\partial y_q}\bigg )\frac{\partial v_{i}}{\partial x_j} dy\,=0\\
\end{align*}
\begin{align}
\Rightarrow &\int_Y \bigg (E_{ijkl}-E_{ijpq}\frac{\partial \chi^{kl}_p}{\partial y_q}\bigg )\frac{\partial v_{i}}{\partial x_j} dy=0  \qquad \text{for k, l = 1 and 2,}
\end{align}
%if $x^{pq}$ satifies
%\begin{equation}
%\int_Y \bigg ( E_{ijkl}-E_{ijpq}\frac{\partial \chi^{kl}_p}{\partial y_q}\bigg ) \frac{\partial v_{1j}}{\partial y_i}\,dy=0 \qquad \text{for k, l = 1 and 2,} 
%\end{equation}
Similarly, substituting equation \eqref{localization} in \eqref{first} gives the homogenized equation.
\begin{align*}
\text{LHS}&=\frac{1}{|Y|}\int_\Omega\int_Y E_{ijkl}(x,y)\bigg (\frac{\partial u_{0k}}{\partial x_l}+\frac{\partial u_{1k}}{\partial y_l}\bigg )\bigg (\frac{\partial v_{0i}}{\partial x_j}\bigg ) dy\,dx\\
&=\frac{1}{|Y|}\int_\Omega\int_Y \bigg (E_{ijkl}\frac{\partial u_{0k}}{\partial x_l}-E_{ijpq}\frac{\partial \chi^{kl}_p}{\partial y_q}\frac{\partial u_{0k}}{\partial x_l}\bigg )\frac{\partial v_{0i}}{\partial x_j} dy\,dx\\
&=\frac{1}{|Y|}\int_\Omega\Bigg \{\int_Y \bigg (E_{ijkl}-E_{ijpq}\frac{\partial \chi^{kl}_p}{\partial y_q}\bigg )dy\Bigg \}\frac{\partial u_{0k}}{\partial x_l}\frac{\partial v_{0i}}{\partial x_j}dx\\
&=\int_\Omega E^H_{ijkl}(x)\frac{\partial u_{0k}}{\partial x_l}\frac{\partial v_{0i}}{\partial x_j}\,dx
\end{align*}
Homogenized equation
\begin{equation}
\int_\Omega E^H_{ijkl}(x)\frac{\partial u_{0k}}{\partial x_l}\frac{\partial v_{0i}}{\partial x_j}\,dx = \int_\Omega\textbf{f}\cdot\textbf{v}_0 dx + \int_{\Gamma_t}\textbf{t}\cdot\textbf{v}_0 ds \qquad \text{for every } \textbf{v}_0
\end{equation}
where $E^H_{ijkl}(x)$ is
\begin{equation}
\boxed{E^H_{ijkl} = \frac{1}{|Y|}\int_Y \bigg (E_{ijkl}-E_{ijpq}\frac{\partial \chi^{kl}_p}{\partial y_q}\bigg ) dy}
\end{equation}
Now, Define
\begin{eqnarray}
a_H(\textbf{u},\textbf{v}) = \int_\Omega E^H_{ijkl}(\textbf{x})\frac{\partial u_k}{\partial x_l}\frac{\partial v_i}{\partial	x_j}\, dx,\\
a_Y(\chi^{kl}, \textbf{v})=\int_Y E_{ijpq}(\textbf{x},\textbf{y})\frac{\partial \chi^{kl}_p}{\partial y_q}\frac{\partial v_i}{\partial y_j}\,dy,\\
L^{kl}_Y(\textbf{v})=\int_Y E_{ijkl} \frac{\partial v_i}{\partial y_j}\, dy
\end{eqnarray}
At microscopic level, we have
\begin{equation}\label{microscopic}
a_Y(\chi^{kl}, \textbf{v}) = L_Y^{kl}(\textbf{v})\qquad \forall \textbf{v}\in \mathcal{U}_Y,
\end{equation}
At macroscopic level, we have
\begin{equation}
a_H(\textbf{u},\textbf{v}) = L(\textbf{v})\qquad \forall \textbf{v}\in \mathcal{U}_0
\end{equation}
where $\mathcal{U}_0$ is homogeneous case of $\mathcal{U}$, i.e., $\textbf{g}=0$.
\subsection{Implementation 2D Homogenization}
Basic homogenization equation, 
\begin{equation}
u_{1i}(\textbf{x},\textbf{y}) = -\chi^{pq}_i\frac{\partial u_{0p}(\textbf{x})}{\partial x_q}
\end{equation}
Solve $\chi^{kl}_p$ from:
\begin{equation}
\int_Y \bigg ( E_{ijkl} - E_{ijpq}\frac{\partial\chi^{kl}_p}{\partial y_q}\bigg ) \frac{\partial v_{1i}}{\partial y_j}\,dy = 0
\end{equation}
Compute:
\begin{equation}\label{homoE}
E_{ijkl}^H=\frac{1}{|Y|}\int_Y\bigg (E_{ijkl} - E_{ijpq}\frac{\partial\chi^{kl}_p}{\partial y_q}\bigg )\,dy
\end{equation}
\subsection{Examples}
Consider: k=1, l=1
\begin{equation}\label{2drhs}
\begin{split}
\int_Y E_{ijkl}\frac{\partial v_i}{\partial y_j}\, dy &=\int_Y E_{ij11}\frac{\partial v_i}{\partial y_j}\,dy\\
&=\int_Y\bigg (E_{1111}\frac{\partial v_1}{\partial y_1}+E_{2211}\frac{\partial v_2}{\partial y_2}\bigg )\, dy
\end{split}
\end{equation}
\begin{equation}\label{2dlhs}
\begin{split}
\int_Y E_{ijpq}\frac{\partial \chi^{kl}_p}{\partial y_q}\frac{\partial v_i}{\partial y_j}\,dy &=\int_Y E_{ijpq}\frac{\partial \chi^{11}_p}{\partial y_q}\frac{\partial v_i}{\partial y_j}\,dy\\
&=\int_Y \bigg \{ E_{11pq}\frac{\partial \chi^{11}_p}{\partial y_q}\frac{\partial v_1}{\partial y_1} + E_{12pq}\frac{\partial \chi^{11}_p}{\partial y_q}\frac{\partial v_1}{\partial y_2}\\
&\qquad\qquad + E_{21pq}\frac{\partial \chi^{11}_p}{\partial y_q}\frac{\partial v_2}{\partial y_1} + E_{22pq}\frac{\partial \chi^{11}_p}{\partial y_q}\frac{\partial v_2}{\partial y_2}\bigg \}\,dy\\
&=\int_Y \bigg \{\bigg ( E_{1111}\frac{\partial \chi^{11}_1}{\partial y_1} + E_{1112}\frac{\partial \chi^{11}_1}{\partial y_2} + E_{1121}\frac{\partial \chi^{11}_2}{\partial y_1} + E_{1122}\frac{\partial \chi^{11}_2}{\partial y_2}\bigg )\frac{\partial v_1}{\partial y_1}\\
&\qquad + \bigg (E_{1211}\frac{\partial \chi^{11}_1}{\partial y_1} + E_{1212}\frac{\partial \chi^{11}_1}{\partial y_2} + E_{1221}\frac{\partial \chi^{11}_2}{\partial y_1} + E_{1222}\frac{\partial \chi^{11}_2}{\partial y_2}\bigg )\frac{\partial v_1}{\partial y_2}\\
&\qquad + \bigg (E_{2111}\frac{\partial \chi^{11}_1}{\partial y_1} + E_{2112}\frac{\partial \chi^{11}_1}{\partial y_2} + E_{2121}\frac{\partial \chi^{11}_2}{\partial y_1} + E_{2122}\frac{\partial \chi^{11}_2}{\partial y_2}\bigg )\frac{\partial v_2}{\partial y_1}\\
&\qquad + \bigg (E_{2211}\frac{\partial \chi^{11}_1}{\partial y_1} + E_{2212}\frac{\partial \chi^{11}_1}{\partial y_2} + E_{2221}\frac{\partial \chi^{11}_2}{\partial y_1} + E_{2222}\frac{\partial \chi^{11}_2}{\partial y_2}\bigg )\frac{\partial v_2}{\partial y_2}\bigg \}\,dy\\
&=\int_Y \bigg \{\bigg ( E_{1111}\frac{\partial \chi^{11}_1}{\partial y_1} + \cancelto{0}{E_{1112}\frac{\partial \chi^{11}_1}{\partial y_2}} + 
\cancelto{0}{E_{1121}\frac{\partial \chi^{11}_2}{\partial y_1}} + 
E_{1122}\frac{\partial \chi^{11}_2}{\partial y_2}\bigg )\frac{\partial v_1}{\partial y_1}\\
&\qquad + \bigg (\cancelto{0}{E_{1211}\frac{\partial \chi^{11}_1}{\partial y_1}} + E_{1212}\frac{\partial \chi^{11}_1}{\partial y_2} + E_{1221}\frac{\partial \chi^{11}_2}{\partial y_1} + \cancelto{0}{E_{1222}\frac{\partial \chi^{11}_2}{\partial y_2}}\bigg )\bigg (\frac{\partial v_1}{\partial y_2} +\frac{\partial v_2}{\partial y_1}\bigg )\\
&\qquad + \bigg (E_{2211}\frac{\partial \chi^{11}_1}{\partial y_1} + \cancelto{0}{E_{2212}\frac{\partial \chi^{11}_1}{\partial y_2}} + \cancelto{0}{E_{2221}\frac{\partial \chi^{11}_2}{\partial y_1}} + E_{2222}\frac{\partial \chi^{11}_2}{\partial y_2}\bigg )\frac{\partial v_2}{\partial y_2}\bigg \}\,dy\\
&=\int_Y \bigg \{\bigg ( E_{1111}\frac{\partial \chi^{11}_1}{\partial y_1} + E_{1122}\frac{\partial \chi^{11}_2}{\partial y_2}\bigg )\frac{\partial v_1}{\partial y_1}\\
&\qquad + E_{1212}\bigg (\frac{\partial \chi^{11}_1}{\partial y_2} + \frac{\partial \chi^{11}_2}{\partial y_1}\bigg )\bigg (\frac{\partial v_1}{\partial y_2} +\frac{\partial v_2}{\partial y_1}\bigg )\\
&\qquad + \bigg (E_{2211}\frac{\partial \chi^{11}_1}{\partial y_1} + E_{2222}\frac{\partial \chi^{11}_2}{\partial y_2}\bigg )\frac{\partial v_2}{\partial y_2}\bigg \}\,dy\\
\end{split}
\end{equation}
Therefore, using equations \eqref{microscopic} , \eqref{2dlhs} and \eqref{2drhs} for k=1, l=1 we have:
\begin{equation}
\begin{split}
&\int_Y \bigg \{\bigg ( E_{1111}\frac{\partial \chi^{11}_1}{\partial y_1} + E_{1122}\frac{\partial \chi^{11}_2}{\partial y_2}\bigg )\frac{\partial v_1}{\partial y_1}\\
&\qquad + E_{1212}\bigg (\frac{\partial \chi^{11}_1}{\partial y_2} + \frac{\partial \chi^{11}_2}{\partial y_1}\bigg )\bigg (\frac{\partial v_1}{\partial y_2} +\frac{\partial v_2}{\partial y_1}\bigg )\\
&\qquad + \bigg (E_{2211}\frac{\partial \chi^{11}_1}{\partial y_1} + E_{2222}\frac{\partial \chi^{11}_2}{\partial y_2}\bigg )\frac{\partial v_2}{\partial y_2}\bigg \}\,dy = \\
&\int_Y\bigg (E_{1111}\frac{\partial v_1}{\partial y_1}+E_{2211}\frac{\partial v_2}{\partial y_2}\bigg )\, dy
\end{split}
\end{equation}
From equation \eqref{homoE}, we can write
\begin{eqnarray}
E^H_{1111} = &\frac{1}{|Y|}\int_Y\bigg ( E_{1111} - E_{1111}\frac{\partial \chi^{11}_1}{\partial y_1} - E_{1122}\frac{\partial \chi^{11}_2}{\partial y_2}\bigg )\,dy\\
E^H_{2211} = &\frac{1}{|Y|}\int_Y\bigg ( E_{2211} - E_{2211}\frac{\partial \chi^{11}_1}{\partial y_1} - E_{2222}\frac{\partial \chi^{11}_2}{\partial y_2}\bigg )\,dy\\
E^H_{1211} = &-\frac{1}{|Y|}\int_Y\bigg (E_{1212}\frac{\partial \chi^{11}_1}{\partial y_2} + E_{1221}\frac{\partial \chi^{11}_2}{\partial y_1}\bigg )\,dy
\end{eqnarray}
Let $\chi^{11}_1 = \Phi_1, \chi^{11}_2 = \Phi_2$ and $E_{1111} = D_{11}, E_{2222} = D_{22}, E_{1212} = D_{66}, E_{1122} = E_{2211} = D_{12}$
\begin{equation}\label{Dformat}
\begin{split}
&\int_Y \bigg \{\bigg ( D_{11}\frac{\partial \Phi^{11}_1}{\partial y_1} + D_{12}\frac{\partial \Phi^{11}_2}{\partial y_2}\bigg )\frac{\partial v_1}{\partial y_1}\\
&\qquad + D_{66}\bigg (\frac{\partial \Phi^{11}_1}{\partial y_2} + \frac{\partial \Phi^{11}_2}{\partial y_1}\bigg )\bigg (\frac{\partial v_1}{\partial y_2} +\frac{\partial v_2}{\partial y_1}\bigg )\\
&\qquad + \bigg (D_{12}\frac{\partial \Phi^{11}_1}{\partial y_1} + D_{22}\frac{\partial \Phi^{11}_2}{\partial y_2}\bigg )\frac{\partial v_2}{\partial y_2}\bigg \}\,dy = \\
&\int_Y\bigg (D_{11}\frac{\partial v_1}{\partial y_1}+D_{12}\frac{\partial v_2}{\partial y_2}\bigg )\, dy
\end{split}
\end{equation}
Also, 
\begin{equation}\label{Dhomo}
D^H_{11} = \frac{1}{|Y|}\int_Y\bigg (D_{11}-D{11}\frac{\partial\Phi_1}{\partial y_1} - D_{12}\frac{\partial \Phi_2}{\partial y_2}\bigg )\,dy
\end{equation}
Rearranging Eq. \eqref{Dformat}
\begin{equation}
\begin{split}
&\int_Y\bigg\{\frac{\partial v_1}{\partial y_1}\quad\frac{\partial v_2}{\partial y_2}\quad\frac{\partial v_1}{\partial y_2}+\frac{\partial v_2}{\partial y_1}\bigg\}
\begin{bmatrix}
D_{11}	& D_{12} & 	0 \\
D_{12}	& D_{22} & 	0 \\
0		& 0		&	D66
\end{bmatrix}\\
&\qquad\times
\begin{bmatrix}
\frac{\partial \Phi_1}{\partial y_1}\\
\frac{\partial \Phi_2}{\partial y_2}\\
\frac{\partial \Phi_1}{\partial y_2}+\frac{\partial \Phi_2}{\partial y_1}
\end{bmatrix}\,dY\\
&=\int_Y\bigg\{\frac{\partial v_1}{\partial y_1}\quad\frac{\partial v_2}{\partial y_2}\quad\frac{\partial v_1}{\partial y_2}+\frac{\partial v_2}{\partial y_1}\bigg\}
\begin{bmatrix}
D_{11}\\
D_{12}\\
0
\end{bmatrix}\,dY
\end{split}
\end{equation}
Let us define
\begin{equation}
\textbf{b} = \begin{bmatrix}
\frac{\partial}{\partial y_1} & 0\\
0 & \frac{\partial}{\partial y_2}\\
\frac{\partial}{\partial y_1} &\frac{\partial}{\partial y_2}\\
\end{bmatrix}
\end{equation}
and 
\begin{equation}
\textbf{D} = \begin{bmatrix}
\textbf{d}_1 & \textbf{d}_2 & \textbf{d}_3\\
\end{bmatrix}
\end{equation}
Then Eq \eqref{Dformat}, can be written as
\begin{equation}
\int_Y \textbf{v}^T \textbf{b}^T \textbf{D} \textbf{b}\Phi\,dY = \int_Y\textbf{v}^T\textbf{b}^T\textbf{d}_1 \qquad \forall \textbf{v}\in\textbf{V}_Y
\end{equation}
and eq. \eqref{Dhomo} becomes:
\begin{equation}
\boxed{D^H_{11} = \frac{1}{|Y|}\int_Y\bigg (D_{11}-\textbf{d}_1^T\textbf{b}\Phi\bigg )\,dy
}
\end{equation}

\section{Results}
\cite{first}
\cite{WEBSITE:1}


\section{References}
\bibliographystyle{plain}
\bibliography{bib}
\end{document}